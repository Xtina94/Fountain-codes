\subsubsection{A comparison with BER bounds on raptor codes}
Theorem 1: Consider a Raptor code with parameters omega(x),
mu(x), KL, K, Q, r and expanding coefficient etha. Its upper
BER bound under ML decoding over a fast Rayleigh fading
channel is Pb leq min{1, Raptor
U
}, where Raptor
U is given by\\

Consider a Raptor code with parameters omega(x),
mu(x), KL, K, Q, r and expanding coefficient etha. Its lower
BER bound under ML decoding over a fast Rayleigh fading
channel is Pb geq max{0, Raptor
L
}, where Raptor
L is given by

\cri{simulazioni}
In this section, we present numerical and simulation results
for the LT-based DNC scheme and the proposed Raptor-based
DNC scheme over quasi-static Rayleigh fading channels. Due
to the time-varying channels of the wireless network, within
each transmission round, a Raptor code is generated on-the-fly
to match the instantaneous network topology. Thus, the BER
performance of an ensemble of codes is analyzed.We consider
the case of K = 100, KL = 50 and 98, which correspond to
the pre-code rate r = 0.5 and 0.98, respectively. We evaluate
the upper and lower bounds on the bit error probability under
ML decoding by using the
