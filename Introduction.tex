\section{Introduction}
Multicast or broadcast transmission is an immediate solution for software companies that intend to spread new software over the Internet directed towards million of users. This is one of the several applications that this kind of transmissions can have nowadays, and this is motivated by the fact that currently there is an incresing proliferation of applications that must reliably distribute huge amount of data. For the nature of the problem, these transmissions have to be fully reliable, support a vast number of receivers and have a low network overhead. All these requirements together imply that retransmission of lost or corrupted packets is not feasible, as the clients requests for retransmission can rapidly overwhelm the server, with consequent \textit{feedback implosion}. \cite{Byers} These problems have led researchers to focus more on the application of erasure codes in \gls{fec} techniques for reliable multicast. Erasure codes are centered around the simple principle that the sender can transmit the $k$ packets, which we could represent the original source data with, together with additional redundant packets that can subsequently be used to recover lost source data at the receiver. This receiver can reconstruct the original source data once it has received a sufficient number of packets and a big benefit in this is that different receivers can recover different lost packets using the same redundant data.

A good application of this concept can be seen in the figure of \textit{digital fountain codes}, a better implementation than the one involving Reed-Solomon codes.\cite{Byers}
