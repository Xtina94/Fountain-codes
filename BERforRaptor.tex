\section{BER bounds for Raptor codes in Rayleigh fading channels}
Finally, another notable result is depicted in \cite{Yue2013} where Yue \textit{et al.} derive a lower bound for the LT-based \gls{dnc} scheme over Rayleigh fading channels and upper/lower bounds for the \gls{ber} in the Raptor based version, in which an inner LT code concatenates a conventional outer code to reduce the error floor.

The scenario of the paper provides a \gls{wsn} with multiple source and relay nodes and a single sink. The main difference with the scenario depicted in \autoref{sec:relayBER} is that here the relay nodes must be of two types: a precoding relay group and an LT-coding relay group. Finally, in this case the decoding is performed throught the \gls{ml} criterium.

\subsection{Lower BER bound for LT scheme}
In \cite{Yue2013} the authors first find a lower bound for the LT-based \gls{dnc} scheme with \gls{ml} decoding over Rayleigh fading channel, starting from the information sequence error probability
\begin{equation}
  P_e = P_r\biggl(\bigcup\limits_{\mathbf{e}:\omega(\mathbf{e})\neq 0}\hat{\mathbf{m} }= \mathbf{e}\biggr)
\end{equation}
In this case, $\hat{\mathbf{m}}$ is the estimated information sequence and $\omega(\cdot)$ the Hamming weight as usual. This probability is given by the probability that only on e symbol in the decoded sequence is recovered uncorrectly. Inside this value, the part representing the Rayleigh fading channel is given by $P_r(\mathbf{v}\rightarrow \mathbf{v}'|\hbar)$, which is the conditional probability that the decoded codeword is equal to $\mathbf{v}'$ give the channel fading coefficient $\hbar$.
