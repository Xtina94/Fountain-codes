\section{An evolution of LT fountain codes: Raptor codes}
\label{sec:raptor}
For many applications, there is then the necessity to construct universal fountain codes which have a fast encoding algorithm and for which the average weight of an output symbol is a constant. Such a class of fountain codes is well seen and described in \cite{Shokrollahi2006} and has the name of \textit{Raptor codes}. Previous results show how LT-codes cannot be encoded with constant cost if the number of collected output symbols is close to the number of input symbols. As a matter of fact, the decoding graph for LT-codes needs to have a number of edges in the order of $Nlog(N)$ ($N =$ is again the number of input symbols) so as to make sure that all the input nodes are recovered with high probability. Here Raptor codes come into play relaxing this condition and requiring that only a \textit{constant fraction} of the input symbols be recoverable. There is a way around the potential problems emerging from this, in particular to the one stating that it is not enough if we recover only a constant fraction of the input symbols, all of them must be recovered \cite{Luby}.

The author addresses the issue encoding the input symbols with the use of a traditional erasure correcting code and later with the application of an appropriate LT-code to the new set of symbols in such a way that the traditional code is capable of recovering all the input symbols even in the case of a fixed fraction of erasures. This is why often Raptor codes are briefly identified as the combination of an LT-fountain code with an appropriate precode \cite{Etesami2006}.

This class of codes can thus be formally defined in this way: let $\mathcal{C}$ be a linear code of block length $n$ and dimension $K$, and let $\Omega(x)$ be a degree distribution; a \textit{Raptor code} with parameters $(K,\mathcal{C},\Omega(x))$ is an LT-code with distribution $\Omega(x)$ on $n$ symbols, representing the coordinates of codewords in $\mathcal{C}$. The code $\mathcal{C}$ is called the \textit{pre-code} of the Raptor Code, while the input symbols of a Raptor code are the $K$ symbols used to construct the codeword in $\mathcal{C}$ consisting of $n$ \textit{intermediate symbols}.
